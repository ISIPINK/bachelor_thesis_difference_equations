\documentclass[a4paper,12pt]{article}
\usepackage[utf8]{inputenc}
\usepackage{amsmath}
\usepackage{amsfonts}
\usepackage{amssymb}
\usepackage[dutch]{babel}
\usepackage{amsthm}
\usepackage{csquotes}
\usepackage[
backend=biber,
style=alphabetic,
]{biblatex}


\setlength{\textwidth}{15.0cm}
\setlength{\textheight}{24.0cm}
\setlength{\topmargin}{0cm}
\setlength{\headsep}{0cm}
\setlength{\headheight}{0cm}
\pagestyle{plain}
\usepackage[dvips]{epsfig}
\usepackage{tikz}
\usepackage[margin=1in]{geometry}


\addbibresource{differentievergelijkingen.bib} 

\setlength{\parindent}{0pt}

%theorems and enviroments

\newtheoremstyle{break}{4pt}{4pt}{}{}{\bfseries}{\vspace{2 pt}}{\newline}{}

\theoremstyle{break}
\newtheorem{theorem}{Stelling}[subsection]
\newtheorem{lemma}[theorem]{Lemma}
\newtheorem{definition}[theorem]{Definitie}
\newtheorem{conjecture}[theorem]{Vermoeden}
\newtheorem{remark}[theorem]{Opmerking}
\newtheorem{example}[theorem]{Voorbeeld}
\newtheorem{corollary}[theorem]{Gevolg}
\newtheorem{notation}[theorem]{Notatie}



\title{Differentievergelijkingen}
\author{Isidoor Pinillo Esquivel}

\begin{document}

\begin{titlepage}
    \begin{center}
        \resizebox{3cm}{!}{\includegraphics{./vert2_kl_01.eps}}
        \ \
        \ \\
        \ \\
        \ \\
        \ \\
        \ \\
        \ \\
        \ \\
        \ \\
        \ \\
        \ \\
        \ \\
        \Large{K{\sc aleidoscopisch projectwerk}}
        \ \\
        \ \\
        \ \\
        \huge{\bf{\em Differentievergelijkingen}}
        \ \\
        \ \\
        \ \\
        \ \\
        \ \\
        \ \\
        \normalsize
        Auteur: {\em Isidoor Pinillo Esquivel}\\
        \ \\
        \ \\
        Begeleider: {\em Werner Peeters}\\
        \ \\
        \ \\
        \ \\
        \ \\
        \ \\
        \ \\
        \ \\
        \ \\

        A{\sc cademiejaar 2021-2022}

    \end{center}
\end{titlepage}

\begin{abstract}
    In deze paper introduceren we lineaire differentievergelijkingen en lineaire operatoren.
    Verder geven we vormen van de algemene oplossing van lineaire differentievergelijkingen door
    companion operatoren (companion matrix) en operatorfactorisatie te gebruiken.
\end{abstract}


\section{Basisbegrippen en notatie}
In deze sectie worden enkele basisbegrippen en algemene notaties ingevoerd. Ter verduidelijking worden er ook een aantal bijhorende voorbeelden gegeven.


% monoïde had ik nodig voor het stuk van lineair onafhankelijkheid
\begin{notation}
    Vanaf nu zal altijd
    \begin{itemize}
        \item  $K$ een willekeurige verzameling zijn.
        \item  $M$ een monoïde zijn.
              % \item  $M^*$ alle inverteerbare elementen van $M$ zijn.
        \item  $Z(M)$ het centrum van $M$ zijn.
        \item  $G$ een groep zijn.
        \item  $A$ een commutatieve/abelse groep zijn.
        \item  $F$ een lichaam zijn.
        \item  $\text{Fun}(K,K) := \{f:K \rightarrow K | f \text{ is een functie van K naar K}\}$.
        \item  $\text{End}(G) := \{f \in \text{Fun}(G,G)| \forall x,y \in G: f(x+y)=f(x)+f(y) \}$.
        \item  $\text{Aut}(G) := \{f \in\text{End}(G)| f \text{ inverteerbaar} \}$.
    \end{itemize}
\end{notation}

\subsection{Rijen en differentievergelijkingen}
Rijen en differentievergelijkingen zullen overal opduiken in deze paper en zijn een eenvoudig beginpunt.
\begin{definition}[rij]
    Een rij is een afbeelding uit $\mathbb{Z}$.
\end{definition}

\begin{remark}
    We hadden rijen ook kunnen definiëren als enkel een afbeelding uit $\mathbb{N}$. Meeste resultaten in deze
    paper zullen ook een versie met deze alternatieve definitie hebben.
    % en is gelaten als een oefening voor de lezer
\end{remark}

% ik heb gekozen om rijen uit Z want anders kom ik later in de problemen met het definiëren van operatoren,
% ik ben daar op een harde manier achter gekomen :P
\begin{notation}[rij]
    Noteer alle rijen naar $K$ als $(K)_{\mathbb{Z}}$ en
    een rij als $(f_i) \in (K)_{\mathbb{Z}}$ waar bijna overal $i \in \mathbb{Z}$ impliciet zal zijn.
\end{notation}



\begin{example}[rij van Fibonacci] %WP Het is Fibonacci, niet Fibonacchi (gefixt)
    Een bekende rij van getallen is die van Fibonacci ($f_i$). Deze rij wordt gedefinieerd aan de hand
    van zijn eerste twee elementen, $f_0$ en $f_1$, respectievelijk gelijk aan 0 en 1, waarna elk volgend element gelijk
    is aan de som van de twee voorgaande, en elk vorig element gelijk is aan het verschil. Of formeel:
    $$
        f_0 = 0, f_1 = 1, f_{n+2} = f_{n+1}+f_{n}, \forall n \in \mathbb{Z}.
    $$

    \begin{table}[ht]
        \centering
        \begin{tabular}{|c|c|c|c|c|c|c|c|c|c|c|} \hline
            ... & $f_{-2}$ & $f_{-1}$ & $f_0$ & $f_1$ & $f_2$ & $f_3$ & $f_4$ & $f_5$ & $f_6$ & ... \\ \hline
            ... & $-1$     & $1$      & $0$   & $1$   & $1$   & $2$   & $3$   & $5$   & $8$   & ... \\ \hline
        \end{tabular}
    \end{table}
\end{example}

De Fibonacci rij uit vorig voorbeeld wordt geconstrueerd door een relatie in de rij. Een type van
relaties in getalrijen zijn differentievergelijkingen.

\begin{definition}[differentievergelijking]
    Een differentievergelijking $\phi$ van orde $k \in \mathbb{N}$ is een rij van relaties binnen een andere rij $(f_i)$
    van volgende vorm:
    $$
        \phi_n((f_j)_{j = n}^{n+k}) = 0, \forall n \in \mathbb{Z}  \text{ met } \phi_i:K^{k+1} \rightarrow K.
    $$
\end{definition}

\begin{remark}
    Aangezien voorgaande definitie geen expliciete differentie bevat, wat later aan bod zal komen
    (zie (\ref{defdiff})), wordt hiernaar ook wel verwezen als recursierelatie.
\end{remark}

\begin{example}[Fibonacci]
    De rij van Fibonacci $(f_i)$ voldoet aan
    $$
        \phi_n((f_j)_{j = n}^{n+k}) = 0, \forall n \in \mathbb{Z} \text{ met } \phi_i:\mathbb{R}^{3} \rightarrow \mathbb{R}:(f_n,f_{n+1},f_{n+2}) \rightarrow f_{n+2}-f_{n+1}-f_n
    $$
    Wat equivalent is met
    $$
        f_{n+2} = f_{n+1}+f_{n}, \forall n \in \mathbb{Z}.
    $$


\end{example}

\begin{definition}[oplossing van een differentievergelijking]
    $(f_i) \in (K)_{\mathbb{Z}}$ is een oplossing van een differentievergelijking $\phi$
    als en slechts als $(f_i)$ aan $\phi$ voldoet.
\end{definition}


%WP: reëel getal, niet reël (gefixt)

\subsection{Operatoren}
Operatoren zullen de notatie significant verbeteren en de grond leggen voor elegantere bewijzen. De naam
''differentievergelijkingen'' komt van differentie operator.

% \cite{boole2008calculus}  wordt dit ook besproken

\begin{notation}
    \label{notoperatoren}
    Om notatie te vereenvoudigen zal vanaf nu
    \begin{itemize}
        \item de index $i$ impliceert een rij.
        \item ronde haakjes duiden aan dat een operator op een rij inwerkt, anders is er sprake van compositie.
        \item constante rijen zullen duidelijk zijn uit hun context, meestal door een ontbrekende index $i$.
        \item bewerkingen tussen rijen gebeuren elementwijs (ook compositie).
        \item functies werken elementwijs op rijen.
        \item matrixoperatorenvermenigvuldiging werken via normale matrixvermenigvuldiging.
    \end{itemize}

\end{notation}

%ik ga achter af alle operatoren die ik ni nodig in comments zetten of steek ik die in een bijlage
\begin{definition}[$E , \Delta , |_k$]
    \label{defdiff}
    % L is de lagg operator, E is de shift operator, $\Delta$ en $\nabla$ zijn differentieoperatoren en $|_k$ is evaluatie in $k$ operator.
    E is de shift operator, $\Delta$ is de differentieoperator en $|_k$ is de ''evaluatie in $k$'' operator.
    \begin{align*}
        % L      & :(K)_{\mathbb{Z}} \rightarrow (K)_{\mathbb{Z}} : (f_i) \rightarrow (f_{i-1})                                                             \\
        E      & :(K)_{\mathbb{Z}} \rightarrow (K)_{\mathbb{Z}} : (f_i) \rightarrow (f_{i+1})       \\
        \Delta & :(G)_{\mathbb{Z}} \rightarrow (G)_{\mathbb{Z}} : (f_i) \rightarrow (f_{i+1}-f_{i}) \\
        % \nabla & :(G)_{\mathbb{Z}} \rightarrow (G)_{\mathbb{Z}} : (f_i) \rightarrow (f_{i}-f_{i-1})                                                       \\
        |_k    & :(K)_{\mathbb{Z}} \rightarrow (K)_{\mathbb{Z}} : (f_i) \rightarrow (f_k)
    \end{align*}
\end{definition}


\begin{lemma}
    $E ,  |_k$ bewaren de elementwijze optelling in $(G)_{\mathbb{Z}}$.
\end{lemma}

\begin{proof}
    Verificatie.
\end{proof}

\begin{definition}[bijna ring]
    Een bijna ring is een ring dat aan minder axioma's voldoet. $R$ is een bijna ring als en slechts als
    \begin{itemize}
        \item R een groep is voor de optelling.
        \item de multiplicatie op R associatief is.
        \item de multiplicatie links en rechts distributief is in R.
    \end{itemize}
\end{definition}

\begin{theorem}
    Alle operatoren die de elementwijze optelling in $(G)_{\mathbb{Z}}$ bewaren, vormen een bijna ring
    met compositie als vermenigvuldiging en de uitgebreide optelling naar operatoren.
\end{theorem}

\begin{proof}
    Verificatie. % of wikipedia  https://en.wikipedia.org/wiki/Near-ring
\end{proof}

\begin{corollary}
    $\forall A,B,C \in End((G)_{\mathbb{Z}}): A(B+C)=AB + AC$.
\end{corollary}

\begin{remark}
    Rechtse distributie is een gevolg van hoe de uitgebreide optelling naar operatoren gedefinieerd is.
\end{remark}


\begin{remark}
    $\Delta$ bewaart alleen de de elementwijze optelling in $(G)_{\mathbb{Z}}$ als deze commutatief is.
    In principe is dit geen probleem want $\Delta = E-1$.
\end{remark}

\begin{lemma}
    $\forall n \in \mathbb{Z}:|_n E = |_{n+1}$ en $ \Delta = E-1$.
\end{lemma}

\begin{proof}
    Verificatie. Uitgeschreven: $\forall f_i \in (G)_{\mathbb{Z}}:$
    \begin{align*}
        (|_n E)(f_i) & = (|_n)(f_{i+1}) \\
                     & = f_{n+1}        \\
                     & = (|_{n+1})(f_i)
    \end{align*}

    \begin{align*}
        (\Delta)(f_i) & = f_{i+1} -f_i \\
                      & = (E-1)(f_i)
    \end{align*}
\end{proof}

\begin{corollary}
    $\forall n \in \mathbb{Z}:|_n \Delta = |_{n+1} - |_n.$
\end{corollary}

\begin{proof}
    Verificatie. Uitgeschreven: $ |_n \Delta = |_n (E-1) = |_n E - |_n = |_{n+1} - |_n.$
\end{proof}


\begin{theorem}
    $\forall g_i \in (\text{Fun}(K,K))_{\mathbb{Z}}: Eg_i = g_{i+1}E.$
\end{theorem}
\begin{proof}
    $\forall f_i \in (K)_{\mathbb{Z}}: E(g_i(f_i)) = g_{i+1}(f_{i+1}) = g_{i+1}(E(f_i)).$
\end{proof}

\begin{corollary}\label{constante functie rijen commuteren met E}
    Dit betekent ook dat constante functie rijen commuteren met $E$.
\end{corollary}


\begin{notation}
    In het geval dat $g_i:M \rightarrow M : m \rightarrow a_i*m$ dan schrijven we gewoon
    $$
        Ea_i = a_{i+1}E
    $$
    dan is de compositie van $a_i$ de elementwijze vermenigvuldiging met $a_i$.
\end{notation}


\begin{definition}[$\Delta^{-1}$]
    $\Delta^{-1}$ is de inverse differentieoperator en wordt gedefinieerd door de volgende eigenschap
    $$
        \Delta^{-1} \Delta = 1 -|_0.
    $$
\end{definition}

\begin{theorem}[goed gedefinieerdheid van $\Delta^{-1}$]
    $\Delta^{-1}$ is goed gedefinieerd en
    \begin{align*}
        \Delta^{-1} : & (G)_{\mathbb{Z}} \rightarrow (G)_{\mathbb{Z}}: (f_n) \rightarrow                              \\
                      & \left\{\begin{aligned}
                                    & \sum_{j=0}^{n-1}{f_{j}} & = & f_{n-1}+f_{n-2}+...+f_1 +f_0     & \text{ als } n > 0 \\
                                    & 0                       &   &                                  & \text{ als } n = 0 \\
                                    & -\sum_{j=n}^{-1}{f_{j}} & = & -f_{n}-f_{n+1}-...-f_{-2}-f_{-1} & \text{ als } n < 0
                               \end{aligned} \right. .
    \end{align*}
\end{theorem}

\begin{proof}
    Verificatie en uniciteit volgt uit (\ref{uniciteitvoorwaarde}).
    Uitgeschreven is dit $\forall f_i \in (G)_{\mathbb{Z}}:$
    \begin{align*}
        (1 -|_0) (f_i)                             & =(\Delta^{-1} \Delta)(f_i) \Leftrightarrow      \\
        f_i - f_0                                  & = (\Delta^{-1})(f_{i+1} - f_i ) \Leftrightarrow \\
        \forall n \in \mathbb{Z}: (|_n)(f_i - f_0) & = (|_n)(\Delta^{-1})(f_{i+1} - f_i )
    \end{align*}
    In overeenstemming met de uiteenvallende definitie van van $\Delta^{-1}$ als functie van $n$, krijgen we hier elementgewijs:\\
    Voor $\forall n>0:$
    \begin{align*}
        f_n - f_0 & =\sum_{j=0}^{n-1}{(f_{j+1} - f_j)}                              \\
                  & = f_{n} -f_{n-1}+f_{n-1}-f_{n-2}+...+f_{2} -f_{1}+ f_{1} -f_{0} \\
                  & = f_n - f_0.
    \end{align*}
    Voor $n=0:$
    $$ f_0 - f_0 = 0.$$
    Voor $\forall n<0:$
    \begin{align*}
        f_n - f_0 & =-\sum_{j=n}^{-1}{(f_{j+1} - f_j)}                          \\
                  & = -(f_0 - f_{-1} + f_{-1} - f_{-2} + ... + f_{n+1} - f_{n}) \\
                  & = -(f_0 - f_{n})                                            \\
                  & = f_n - f_0.
    \end{align*}
\end{proof}



\begin{lemma}
    $
        \Delta \Delta^{-1} = 1.
    $
\end{lemma}

\begin{proof}
    Verificatie. Uitgeschreven :
    \begin{align*}
        1                             & =  \Delta \Delta^{-1} \Leftrightarrow         \\
        \forall n \in \mathbb{Z}: |_n & = |_n \Delta \Delta^{-1}\Leftrightarrow       \\
        \forall n \in \mathbb{Z}: |_n & = (|_{n+1} - |_n) \Delta^{-1} \Leftrightarrow \\
        \forall n \in \mathbb{Z}: |_n & = |_{n+1}\Delta^{-1}  - |_n\Delta^{-1}
    \end{align*}
    $\forall f_i \in (G)_{\mathbb{Z}}:$\\
    Voor $\forall n>0:$
    \begin{align*}
        f_n & = (|_{n+1}\Delta^{-1}  - |_n\Delta^{-1}  )(f_i)            \\
            & =  f_{n}+f_{n-1}+...+f_1 +f_0  - f_0 - f_1 - ... - f_{n-1} \\
            & = f_n.
    \end{align*}
    Voor $n=0:$
    \begin{align*}
        f_0 & = (|_{1}\Delta^{-1}  - |_0\Delta^{-1}  )(f_i) \\
            & =  f_0 - 0                                    \\
            & = f_0.
    \end{align*}
    Voor $n=-1:$
    \begin{align*}
        f_{-1} & = (|_{0}\Delta^{-1}  - |_{-1}\Delta^{-1}  )(f_i) \\
               & =  0 - (-f_{-1})                                 \\
               & =  f_{-1}.
    \end{align*}
    Voor $\forall n<-1:$
    \begin{align*}
        f_n & = (|_{n+1}\Delta^{-1}  - |_n\Delta^{-1}  )(f_i)                                        \\
            & = -f_{n+1}-f_{n+2}-...-f_{-2}-f_{-1} + f_{-1} + f_{-2} + ... + f_{n+2} + f_{n+1} + f_n \\
            & = f_n.
    \end{align*}

\end{proof}


%WP Bedoel je met die laatste operator dat de rij constant f_k wordt? Hier bij deze definitie een voorbeeldje geven zou een hoop verduidelijken.
%Verder zou ik ook vanaf het moment dat er een som of verschil staat, je haken groter maken met \left en \right, en ik zou niet \Sigma en \Pi gebruiken, 
%maar \sum en \prod, al dan niet voorafgegaan door \displaystyle

%isi: f_k is constant die \sum en \prod maakt alles mega groot dus ik heb dat opgesplitsts in 2 defs en nu ziet
% dat er beter uit


\begin{example}[operatoren en Fibonacci]
    Met voorgaande operatoren kunnen we nu, met identieke beginvoorwaarden, de genererende differentievergelijking voor de rij van Fibonacci herschrijven in termen van $E$:
    $$
        (E^2 -E -1)(f_i) =0.
    $$
    Dit kan ook met $\Delta$:
    \begin{align*}
        ((\Delta +1)^2 - (\Delta +1) +1)(f_i) & =0  \\
        (\Delta^2 + \Delta + 1)(f_i)          & =0.
    \end{align*}
    In volgende tafel staan een paar voorbeelden van operatoren werkende op de rij van Fibonacci:

    \begin{table}[ht]
        \centering
        \begin{tabular}{|c||r|r|r|r|r|r|r|r|r|r|r|} \hline
            $i$                & ... & $-2$ & $-1$ & $0$ & $1$ & $2$ & $3$ & $4$ & $5$ & $6$  & ... \\ \hline \hline
            $f_i$              & ... & $-1$ & $1$  & $0$ & $1$ & $1$ & $2$ & $3$ & $5$ & $8$  & ... \\ \hline
            $E(f_i)$           & ... & $1$  & $0$  & $1$ & $1$ & $2$ & $3$ & $5$ & $8$ & $13$ & ... \\ \hline
            $\Delta(f_i)$      & ... & $2$  & $-1$ & $1$ & $0$ & $1$ & $1$ & $2$ & $3$ & $5$  & ... \\ \hline
            $(f_i)|_1$         & ... & $1$  & $1$  & $1$ & $1$ & $1$ & $1$ & $1$ & $1$ & $1$  & ... \\ \hline
            $\Delta^{-1}(f_i)$ & ... & $0$  & $-1$ & $0$ & $0$ & $1$ & $2$ & $4$ & $7$ & $12$ & ... \\ \hline
        \end{tabular}
    \end{table}
\end{example}




% hier zeg ik niet wat a_i en f_i zijn ... maar dit is puur om notatie te demonstreren
\begin{example}[haakjes en operatoren]
    Volgend op (\ref{notoperatoren}) letten we op volgende mogelijke misvattingen:
    \begin{align*}
        E a_i                                       & \neq E(a_i) \Leftrightarrow                 \\
        E(a_i f_i) = a_{i+1}f_{i+1}                 & \neq a_{i+1} f_i = E(a_i) f_i.              \\
        % WP Deze notatie maakt de zaak alvast niet duidelijker. Volgens je eerdere definitie werkt E op een element van (\mathbb{R})_(\mathbb{Z}), 
        % dus weet ik niet goed wat je met E a_i bedoelt. Definieer ook a_i!
        %
        % In vorig voorbeeld is de eerste vergelijking een operator ongelijkheid met het LL vermenigvuldig met $a_i$ en dan een shift
        % en in het RL een vermenigvuldig met de shift van $a_i$. Om te checken dat operatoren gelijk zijn moeten ze op dezelfde manier % WP: opdezelde? (hopelijk gefixt)
        % werken op alle rijen wat duidelijk niet zo is in dit geval. Dit is hetzelfde voorbeeld maar dan met de differentieoperator:
        \Delta a_i                                  & \neq \Delta(a_i) \Leftrightarrow            \\
        \Delta(a_i f_i) = a_{i+1} f_{i+1} - a_i f_i & \neq a_{i+1} f_i - a_i f_i=\Delta(a_i) f_i.
    \end{align*}


\end{example}

\begin{example}[constante uit context]
    Analoog aan voorgaande voorbeeld, en gelet op de notatie van een constante rij, mag $E(f_i) = f_{i+1}$ niet te verwarren zijn met:
    \begin{align*}
        E(f_0) & = f_0     \\
               & \neq f_1.
    \end{align*}

    Aangezien de shift operator inwerkt op $(K)_\mathbb{(Z)}$ en niet op $\mathbb{Z}$, kan $f_0$ enkel wijzen op een constante rij $(f_0)_{i \in \mathbb{Z}}$.
\end{example}

\begin{example}[matrixoperatoren]
    We starten met een eenvoudig voorbeeld:
    \[
        \left[ {
                    \begin{array}{c}
                        E \\
                        1
                    \end{array} } \right]
        (f_i) =
        \left[ {
                    \begin{array}{c}
                        E(f_i) \\
                        1(f_i)
                    \end{array} } \right]
        =
        \left[ {
                    \begin{array}{c}
                        f_{i+1} \\
                        f_i
                    \end{array} } \right].
    \]
    Vervolgens hetzelfde voorbeeld maar dan met een matrixrij:
    \[
        \left[ {
                    \begin{array}{c}
                        E \\
                        1
                    \end{array} } \right]
        \left(\left[ {
            \begin{array}{cc}
                g_i & h_i
            \end{array} } \right]\right) =
        \left[ {
                    \begin{array}{r}
                        E \left(\left[ {
                        \begin{array}{cc}
                                g_i & h_i
                            \end{array} } \right]\right) \\
                        1\left(\left[ {
                            \begin{array}{cc}
                                g_i & h_i
                            \end{array} } \right]\right)
                    \end{array} } \right]
        =
        \left[ {
                    \begin{array}{c}
                        \left[ {
                        \begin{array}{cc}
                                        g_{i+1} & h_{i+1}
                                    \end{array} } \right] \\
                        \left[ {
                                    \begin{array}{cc}
                                        g_i & h_i
                                    \end{array} } \right]
                    \end{array} } \right].
    \]
    Verder hetzelfde voorbeeld zonder haakjes:
    \[
        \left[ {
                    \begin{array}{c}
                        E \\
                        1
                    \end{array} } \right]
        \left[ {
                    \begin{array}{cc}
                        g_i & h_i
                    \end{array} } \right] = Eg_i + h_i = g_{i+1}E + h_i.
    \]
    Verder hetzelfde voorbeeld waar de volgorde van de vermenigvuldiging gewisseld is:
    \[
        \left[ {
                    \begin{array}{cc}
                        g_i & h_i
                    \end{array} } \right]
        \left[ {
                    \begin{array}{c}
                        E \\
                        1
                    \end{array} } \right]
        = g_iE + h_i.
    \]

\end{example}

% uit een oude versie
% % hier kan ik ook diegene weghalen die ik niet nodig heb of mss steek ik dat in een bijlage
% \begin{lemma}[relaties operatoren]
%     Tussen de operatoren zelf bestaan er relaties die te checken zijn.
%     \begin{align*}
%         1             & = LE = EL                   \\
%         0             & = \Delta |_k = \nabla |_k   \\
%         \Delta        & = E -1 = E\nabla = \nabla E \\
%         \nabla        & = 1 -L = L\Delta= \Delta L  \\
%         \Delta \Sigma & =1                          \\
%         \Sigma \Delta & = 1 - |_{0}                 \\
%         |_{k+1}       & = |_k E                     \\
%         |_{k}         & = E |_k  = |_{l} |_k        \\
%         E a_i         & = a_{i+1} E
%     \end{align*}
%     meer relaties zijn te vinden door inductie, binomium van Newton of combinaties te maken.
% \end{lemma}





% De theorie voor differentievergelijkingen loopt voor een groot stuk parallel met die van differentiaalvergelijkingen. Men kan bv
% ODE's oplossen met differentievergelijkingen door deze te discretiseren en dan de limiet te nemen. Dit is te doen door onderligende rijen $(x_i)$ te kiezen,
% waar $\Delta(x_i) \rightarrow 0$. Vervang dan in de diffrentiaalvergelijking $x$ met $(x_i)$, $f$ met $(f_i) \approx (f(x_i))$ %WP differentiaalvergelijking; "waar dat" is geen correct Nederlands, die "dat" moet weg. (gefixt)
% en alle $\frac{d}{dx}$ door $\frac{\Delta}{\Delta (x_i)}$. Los dit voor alle onderliggende rijen op en neem de limiet van de oplossing. 
% %WP met moet door zijn! Tevens is het me niet duidelijk wat je bedoelt met een rij gedeeld door een andere rij!
% %isi: ik heb nu elementwijze bewerkingen tussen rijen.

% \begin{example}
%     Neem
%     $$
%         \frac{df}{dx} = f
%     $$
%     Kies dan onderliggende rijen  $x_i = ti$ met $t \in \mathbb{R}$. Dit geeft %WP onderliggende (gefixt)
%     $$
%         \frac{f_{i+1} - f_{i}}{t} = f_i \Leftrightarrow f_{i} =  (1 + t)^i f_0
%     $$
%     en  $f(x)$ is de limiet hiervan met $x = tn$
%     $$
%         f(x) = \lim_{t \rightarrow \infty}{(1+t)^n f_0 } = f(0) \lim_{n \rightarrow \infty}{ \left(1+\frac{x}{n}\right)^n} = f(0) e^{x} 
%         % WP Haken groter maken met \left en \right (gefixt)
%     $$
% \end{example}

\section{Lineaire differentievergelijkingen}
Een algemene discussie van differentievergelijkingen valt buiten de opzet van dit werk, waardoor we ons hier beperken tot het lineaire geval.
In deze sectie zullen we matrixoperatoren en operatorfactorisatie gebruiken om oplossingen van
differentievergelijkingen te bestuderen en demonstreren we een aantal resultaten met voorbeelden.
Hiervoor definiëren we nog een aantal begrippen.

\begin{definition}[lineaire differentieoperator]
    We noemen $\alpha_i\in (\text{End}(G))_\mathbb{Z}[E]$ met $\text{Deg}(\alpha_i)=k$
    een lineaire differentieoperator van orde/graad $k$.
\end{definition}

\begin{notation}[polynomen in $E$]
    We schrijven polynomen in $E$ met Griekse letters.
\end{notation}

\begin{definition}[(in)homogeen lineaire differentievergelijkingen, homogene/ particuliere oplossingen, forceerterm] %WP differentievergelijkingen is één woord.(gefixt)
    Een differentievergelijking van $f_i \in (G)_{\mathbb{Z}}$ noemt men lineair van orde $k \in \mathbb{N}$ als en slechts als
    deze herschreven kan worden in volgende vorm:
    \begin{align*}
        (\alpha_i)(f_i)                                                   & = g_i \Leftrightarrow \\
        (a^0_iE^k + a^1_i E^{k-1} + a^2_i E^{k-2} + ... + a^k_i E^0)(f_i) & = g_i
    \end{align*}


    %WP: ik ben er niet zeker van dat alle a^j_i verschillend moeten zijn van nul; ik kan me best voorstellen dat er een term in de formule ontbreekt, 
    %en dat het volstaat dat alleen de hoogste ordeterm verschilt van nul.

    %isi: kijk naar de opmerking
    met
    \begin{itemize}
        \item $\alpha_i \in (\text{End}(G))_\mathbb{Z}[E]$ de corresponderende lineaire operator.
        \item $a^j_i \in (\text{End}(G))_\mathbb{Z}, \forall j$ de coëfficiënten.
        \item $g_i \in (G)_{\mathbb{Z}}$ de forceerterm.
    \end{itemize}

    Lineaire differentievergelijkingen worden homogeen genoemd als $g_i = 0$ en anders
    inhomogeen. Oplossingen voor deze types differentievergelijkingen worden respectievelijk homogeen en particulier genoemd.
\end{definition}



\begin{example}[Fibonacci]
    -De rij van Fibonacci is een lineaire differentie vergelijking van de tweede graad, en voldoet aan $(E^2 -E -1)(f_i) =0.$
\end{example}



% ik had hier voor niet de voorwaarde dat de functies optelling moest bewaren
% \begin{example}[rijen van Hailstone]
%     De rijen van Hailstone komt voor in de beruchte Collatz conjecture. Deze zijn gedefinieerd op volgende manier:
%     \begin{align*}
%         f(n)&= \begin{cases}\frac{n}{2} & \text { als } n \equiv 0(\bmod 2) \\ 3 n+1 & \text { als } n \equiv 1(\bmod 2)\end{cases}\\
%         a_{i}&= \begin{cases}a_0 & \text { voor } i=0 \\ f\left(a_{i-1}\right) & \text { voor } i>0\end{cases}.
%     \end{align*}
%     Voor $i> 0$ geldt dus volgende differentievergelijking:
%     $$
%     (E-f)(a_i)=0.
%     $$
% \end{example}

% \begin{remark}
%     In principe kan vorig voorbeeld gedaan worden met alle rijen die gedefinieerd zijn door herhaaldelijke functie evaluaties.    
% \end{remark}


%dit ziet er niet zo overzichtelijk uit mss doe ik daar later iets aan mss
% mss splits ik dit op
\begin{theorem}[uniciteitvoorwaarde]
    \label{uniciteitvoorwaarde}
    Gegeven een lineaire differentievergelijking van orde $k$ met $a^j_i$ als coëfficiënten van $f_i$ waar dat
    $\forall i< l+1 , i> l+k:a_i^0,a_i^k \in \text{Aut}(A)$ en gegeven $k$
    opeenvolgende waarden van $f_i$ $(f_j)_{j=l+1}^{l+k}$ dan is $f_i$ uniek bepaald.
\end{theorem}
\begin{proof}
    2 simpele inductie bewijzen voor $(f_j)_{j< l+1}, $ en $(f_j)_{ j> l+k} $.
\end{proof}

\begin{remark}
    Deze stelling kunnen we opsplitsen in 2 gevallen.
    Dit wil zeggen als we alleen geïnteresseerd waren in $(f_j)_{ j> l+k}$, dat enkel
    $\forall i> l+k:a_i^k \in \text{Aut}(A)$ nodig is voor uniciteit.
\end{remark}

\begin{definition}[uniciteitvoorwaarde]
    $\alpha_i \in (\text{End}(G))_\mathbb{Z}[E]$ van graad $k$ voldoet aan de uniciteitvoorwaarde als en
    slechts als
    $
        a_i^0,a_i^k \in (\text{Aut}(G))_{\mathbb{Z}}.
    $
    Een differentievergelijking voldoet aan de uniciteitvoorwaarde als de corresponderende lineaire operator
    dat doet.
\end{definition}



\begin{lemma}
    \label{lemma unciteit voorwaarde}
    $\alpha_i, \beta_i \in (\text{End}(G))_\mathbb{Z}[E]$ voldoen aan de uniciteitvoorwaarde
    als en slechts als $\alpha_i\beta_i$ aan de uniciteitvoorwaarde voldoet.
\end{lemma}

\begin{proof}
    Verificatie.
\end{proof}

\begin{definition}[annihilator]
    $\beta_i \in (\text{End}(G))_\mathbb{Z}[E]$ is een annihilator van $g_i \in (G)_{\mathbb{Z}}$
    als en slechts als
    $
        (\beta_i)(g_i) = 0.
    $
\end{definition}

\begin{theorem}
    \label{annihilator}
    Als $f_i \in (G)_{\mathbb{Z}}$ voldoet aan een differentievergelijking met een lineaire operator $\alpha_i$
    die aan de uniciteitvoorwaarde voldoet en een forceerterm die een annihilator $\beta_i$ heeft
    die ook aan de uniciteitvoorwaarde voldoet dan
    is $f_i$ uniek bepaald door
    $
        (\beta_i\alpha_i)(f_i) = 0
    $
    en $\text{Deg}(\beta_i\alpha_i)$ opeenvolgende waardes van $f_i$.
\end{theorem}

\begin{proof}
    Uit (\ref{lemma unciteit voorwaarde}) volgt dat $\beta_i\alpha_i$ voldoet aan de uniciteitsvoorwaarde, zodat $f_i$ uniek bepaalt is door $\text{Deg}(\beta_i\alpha_i)$ opeenvolgende waarden van $f_i$
    (\ref{uniciteitvoorwaarde}).
\end{proof}

\begin{example}
    Neem bijvoorbeeld
    $$
        (E-1)(f_i) = 2^i , f_0 =0.
    $$
    Een annihilator van $2^i$ is $E-2$ , om uniciteit te behouden neem extra beginvoorwaarde $f_1 = 1$ wat te
    berekenen is uit vorige relatie. Dus het vorig probleem is equivalent met
    $$
        (E-2)(E-1)(f_i)= (E^2-3E+2)(f_i) = 2^{i+1}-2^{i+1}=0, f_0 =0, f_1=1.
    $$
\end{example}

\begin{definition}[$\prod$]
    $\prod$ is de product operator gedefinieerd op volgende manier:
    $$
        \prod:(\text{Aut}(G))_\mathbb{Z} \rightarrow(\text{Aut}(G))_\mathbb{Z}: f_i \rightarrow s_i
    $$
    waardat $s_i$ de unieke oplossing is van
    \begin{align*}
        (E-f_i)(s_i) & = 0 \Leftrightarrow \\
        s_{i+1}      & = f_i s_i
    \end{align*}
    met $s_0 = 1$.
\end{definition}

\begin{remark}
    Merk op dat bovenstaande vergelijking ook een lineaire differentievergelijking is. $\text{Aut}(G)$
    is een groep met de uitgebreide optelling uit $G$ en de vermenigvuldiging met $f_i$ kan dan bekeken worden
    als een element van $\text{Aut}(\text{Aut}(G))$.
\end{remark}

\begin{remark}
    Als we hierboven in plaats van de producten de optelling hadden gezet, kwamen we terug uit op de definite van
    $\Delta^{-1}$.
\end{remark}

\begin{theorem}[Goed gedefinieerdheid van $\prod$]
    $\prod$ is goed gedefinieerd en
    \begin{align*}
        \prod: & (\text{Aut}(G))_\mathbb{Z} \rightarrow(\text{Aut}(G))_\mathbb{Z}: f_i \rightarrow                      \\
               & \left\{\begin{aligned}
                             & \prod_{j=0}^{n-1}{f_{j}}       & = & f_{n-1}f_{n-2}...f_1 f_0           & \text{ als } n > 0 \\
                             & 1                              &   &                                    & \text{ als } n = 0 \\
                             & (\prod_{j=n}^{-1}{f_{j}})^{-1} & = & (f_{-1}f_{-2}...f_{n+1}f_{n})^{-1} & \text{ als } n < 0
                        \end{aligned} \right. .
    \end{align*}
\end{theorem}

\begin{proof}
    Verificatie. Uniciteit volgt uit \ref{uniciteitvoorwaarde},
    analoog aan goed gedefinieerdheid van $\Delta^{-1}$.
\end{proof}

\begin{example}
    $\prod{f} = f^i.$
\end{example}

\begin{theorem}[variatie op integrerende factor]
    $
        \forall a_i,b_i \in (\text{Aut}(G))_\mathbb{Z}, \exists h_i,s_i\in (\text{Aut}(G))_\mathbb{Z}:
        a_i E + b_i = s_i \Delta h_i.
    $

\end{theorem}
\begin{proof}
    \begin{align*}
        a_i E + b_i & = s_i \Delta h_i        \\
                    & = s_i(E-1)h_i           \\
                    & = s_i E h_i -s_i h_i    \\
                    & = s_i h_{i+1}E -s_i h_i
    \end{align*}
    Door nu LL en RL te vergelijken of van elkaar af te trekken
    is het eenvoudig te zien dat het volgende een voldoende voorwaarde is:
    $$
        \left\{\begin{aligned}
            a_i & = s_i h_{i+1} \\
            b_i & = -s_i h_i
        \end{aligned} \right. \quad \Leftrightarrow
        h_i^{-1} = (-1)^i\prod{\left(b_ia_i^{-1}\right)} h_0^{-1} \text{ en } s_i = - b_i h_i^{-1}.
    $$
    Door $h_0=1$ kiezen hebben we een $h_i, s_i$ geconstrueerd die aan de voorwaarde voldoen.
\end{proof}

% WP: Het is me niet duidelijk wat de stelling hier zegt. 
%Je begint ook in je bewijs met wat je moet bewijzen. 
%Ook staat het hier vol produkttekens zonder indices - dit is niet dadelijk duidelijk!

%isi:stelling zegt dat de operator van die vorm anders geschreven kan worden, 
% het product heeft geen indices omdat het een operator is
% notatie is inderdaad niet duidelijk maar eenvoudiger

\begin{corollary}
    \label{oplo1}
    Als $f_i$ een 1ste orde lineaire differentie vergelijking is, die voldoet aan de uniciteitsvoorwaarde:

    $$
        (a_i^0E + a_i^1)(f_i) = g_i
    $$
    met beginwaarde $f_0$, dan kan $f_i$ formeel geschreven worden als:
    $$
        f_i= h_i^{-1} \left( \Delta^{-1} {\left( s_i^{-1} g_i \right)} + f_0 \right)
    $$
    met $h_i$ en $s_i$ uit vorige stelling.
    %WP: Hier staat een somteken zonder indices!
    %isi: de som heeft geen indices omdat het een operator is
\end{corollary}

\begin{proof}
    \begin{align*}
        (a_i^0E + a_i^1)(f_i)        & = g_i    \Leftrightarrow                                                   \\
        (s_i \Delta h_i)(f_i)        & =g_i     \Leftrightarrow                                                   \\
        s_i \Delta (h_i f_i)         & =g_i     \Leftrightarrow                                                   \\
        \Delta (h_i f_i)             & =s_i^{-1} g_i  \Leftrightarrow                                             \\
        \Delta^{-1} \Delta (h_i f_i) & = \Delta^{-1}  {\left( s_i^{-1} g_i \right)}               \Leftrightarrow \\
        (1 - |_0) (h_i f_i)          & = \Delta^{-1}  {\left( s_i^{-1} g_i \right)}               \Leftrightarrow \\
        h_i f_i                      & = \Delta^{-1}  {\left( s_i^{-1} g_i \right)} + h_0 f_0     \Leftrightarrow \\
        f_i                          & = h_i^{-1}\left( \Delta^{-1}  {\left( s_i^{-1} g_i \right)} + f_0 \right)
    \end{align*}
\end{proof}

\begin{remark}
    $1$ste orde lineaire operatoren die aan de uniciteitvoorwaarde voldoen hebben dus een soort van inverse.
\end{remark}

\begin{remark}
    In het geval dat we ons zouden beperken tot positieve indices, dan hebben we enkel de uniciteitsvoorwaarde nodig in één richting, en vereenvoudigen de definities van
    $\Delta^{-1}, \prod$.
\end{remark}

% hier ben ik niet zo zeker van
% \begin{remark}
%     Als we enkel geïnteresseerd zouden zijn in waarde van een rij met indexen in
%     een richting $(f_j)_{j>l}$ dan is het niet voldoende dat enkel $a_i^0$ inverteerbaar is om een
%     soort gelijke formule te gebruiken.
%     Een deelse oplossing bestaat door opsplitsen in eindige gevallen wanneer $a_i^1$ niet
%     inverteerbaar is maar hier gaan we niet verder op in.
% \end{remark}

% Hier ga ik eigenlijk niet verder op in ...
% \begin{remark}[inverse   $a_i^0E + a_i^1$]
%     We kunnen $f_i$ transformeren naar $t_i = f_i - f_0$ zodat $t_0 =0$ en dan kunnen we spreken over de ''inverse''
%     van  $a_i^0E + a_i^1$. %WP De derde persoon enkelvoud van impliceren is impliceerT, niet impliceerD (gefixt)
%     dit zou dan $h_i^{-1}\sum s_i^{-1}$ zijn.
% \end{remark}

\begin{example}
    In het geval van
    $$
        (E -(i+1))(f_i) = (i+1)!
    $$
    die enkel voldoet aan de uniciteitsvoorwaarde in 1 richting, dan krijgen we voor $h_i$ en $s_i$:
    $$
        h_i = \frac{1}{i!} \text{ en } s_i = (i+1)!
    $$
    Wanneer we dit invullen in (\ref{oplo1}), dan krijgen we
    $$
        f_i = i! \left( \sum {\left( \frac{(i+1)!}{(i+1)!} \right)} + f_0 \right) = i! i + i! f_0
    $$
    hetgeen enkel geldt voor positieve indices.
\end{example}

% \begin{remark}
%     Vorig voorbeeld is uitgekozen voor eenvoudige computatie in het algemeen is de oplossing geen
%     ''mooie'' formule.
% \end{remark}

\begin{remark}
    Tot dusver, hebben we nog geen gebruik gemaakt van het feit dat de coëfficiënten van een differentievergelijking
    de optelling moeten behouden.
\end{remark}




\begin{definition} [$\text{Opl}(\alpha_i, g_i)$]
    Defineer voor $\alpha_i \in (\text{End}(G))_\mathbb{Z}[E], g_i \in (G)_{\mathbb{Z}}:$
    $$
        \text{Opl}(\alpha_i, g_i)  := \left\{f_i \in (G)_{\mathbb{Z}} | (\alpha_i)(f_i)=g_i \right\}.
    $$
\end{definition}

\begin{lemma}
    \label{centrumlemma}
    $\forall \alpha_i \in (Z(\text{End}(G)))_\mathbb{Z}[E], \forall g \in (\text{End}(G))_\mathbb{Z}: \alpha_ig = g\alpha_i.$
\end{lemma}

\begin{proof}
    $g$ commuteert met $E$ (\ref{constante functie rijen commuteren met E}) en $g$ commuteert met de coëfficiënten van
    $\alpha_i$ omdat die in het centrum zitten.
\end{proof}

% de ring van constante rij endomorfismen op $A$ klinkt raar maar ok
\begin{theorem}[structuur $\text{Opl}(\alpha_i,0)$ ]
    Als $ \alpha_i \in (Z(\text{End}(A)))_\mathbb{Z}[E]$ dan is $\text{Opl}(\alpha_i,0)$ een links $R$-module
    met $R \simeq \text{End}(A)$ de ring van constante rij endomorfismen op $A$.
\end{theorem}

\begin{proof}
    $\forall x,y \in \text{Opl}(\alpha_i,0), \forall g \in R:$
    \begin{align*}
        (\alpha_i)(x+y) & = (\alpha_i)(x) + (\alpha_i)(y) = 0 \Rightarrow x+y \in \text{Opl}(\alpha_i,0)       \\
        (\alpha_i)(gx)  & = (\alpha_ig)(x) = (g\alpha_i)(x) = g(0)=0 \Rightarrow gx \in \text{Opl}(\alpha_i,0)
    \end{align*}
    In de laatste lijn wordt (\ref{centrumlemma}) gebruikt.
\end{proof}

\begin{corollary}
    Als $\text{End}(A)$ een lichaam is dan geldt $\forall \alpha_i \in
        (\text{End}(A))_\mathbb{Z}[E]:\text{Opl}(\alpha_i,0)$ is een vectorruimte.
\end{corollary}

\begin{theorem}[structuur $\text{Opl}(\alpha_i,g_i)$ ]
    $\forall \alpha_i \in (\text{End}(A))_\mathbb{Z}[E], \forall g_i \in (A)_{\mathbb{Z}}, \forall f_P \in
        \text{Opl}(\alpha_i,g_i):$
    $$
        \text{Opl}(\alpha_i,g_i) = f_P + \text{Opl}(\alpha_i,0)
    $$
\end{theorem}

\begin{proof}
    Verificatie, analoog aan het resultaat in differentiaalvergelijkingen.
\end{proof}


% dit is een ding uit vorige versies kan mss nog nuttig zijn achter af
% \begin{corollary}[oplossing linearie differentievergelijkingen met constante coëfficiënten]
%     Lineaire differentievergelijkingen met constante coëfficiënten oplossen reduceert naar factoriseren
%     van het operatorpolynoom.
% \end{corollary}

% \begin{proof}
%     Alle lineaire differentievergelijkingen met constante coëfficiënten kunnen omgezet worden naar volgende vorm:
%     $$
%         Of_i = g_i \text{ met } O \in \mathbb{R}_0[E]
%     $$
%     %WP: Ik heb geen idee wat je bedoelt met \mathbb{R}_0[E]
%     %isi: polynomen in E met coef in \mathbb{R}_0
%     Aangezien constantes commuteren met $E$ is $\mathbb{R}_0[E] \simeq  \mathbb{R}_0[X]$. Als we hetzelfde resultaat in $\mathbb{C}$
%     kunnen aantonen volgt het gestelde want dan construeren we een oplossing binnen $\mathbb{C}$ en dit is voldoende want de oplossing
%     voor een operator in $\mathbb{R}_0[E]$ is altijd binnen $(\mathbb{R})_{\mathbb{Z}}$ zelfs als de formule voor de oplossing
%     gebruik maakt van $\mathbb{C}$. In $\mathbb{C}$ is het resultaat triviaal want daar factoriseren polynomen in eerste graad polynomen
%     en voor operatoren van deze vorm kunnen we een inverse construeren wat de inverse van heel de polynoom construeerbaar maakt.
% \end{proof}

% \begin{remark}
%     Een andere manier om dit probleem te benaderen is om een gok te doen met parameters en aan te tonen dat de oplossing
%     compleet is.
% \end{remark}


% concept van reductie van coeficienten uit vorige versies
% Nu gaan we verder met het idee van het factoriseren van de operator in operatoren
% die ''geinverteerd'' kunnen worden.

% \begin{lemma}
%     $E^2 +a_i E +b_i$ factoriseren kan gereduceerd worden naar $E^2 +c_i E - 1$ factoriseren.
% \end{lemma}

% % WP: Dit is wiskundig niet eens een exact geformuleerde bewering. Wat is ``even moeilijk''?
% %isi: zal ik anders verwoorden
% \begin{proof}
%     \begin{align*}
%         E^2 +a_i E +b_i & = (E^2 +a_i E +b_i)u_i u^{-1}_i                                            \\
%                         & = (E^2 u_i +a_i E u_i + u_i b_i)u^{-1}_i                                   \\
%                         & = (u_{i+2} E^2  +a_i u_{i+1}E  + u_i b_i)u^{-1}_i                          \\
%                         & = u_{i+2}( E^2  +a_i u_{i+1}u_{i+2}^{-1}E  + u_i u_{i+2}^{-1} b_i)u^{-1}_i
%     \end{align*}
%     Kies $u_i$ zodat
%     $$
%         u_i u_{i+2}^{-1} b_i = -1 \Leftrightarrow u_{i+2} = -b_i u_i
%     $$
%     Waarvoor expliciete oplossingen voor bestaan.
%     Vervang $a_i u_{i+1}u_{i+2}^{-1}$ met $c_i$ dit geeft
%     $$
%         E^2 +a_i E +b_i =u_{i+2}( E^2  + c_iE  -1 )u^{-1}_i
%     $$
%     tegenovergestelde richting is triviaal.
% \end{proof}

% \begin{lemma}
%     Als $a_i$ en $b_i$ periodiek zijn kan in vorig bewijs $u_i$ zo gekozen worden misschien in $\mathbb{C}$ zodat
%     $c_i$ ook periodiek zal zijn.
% \end{lemma}
% %WP: Wat is periodiek?
% %isi: moet ik nog fixen

% \begin{proof}
%     zonder bewijs
% \end{proof}


% \begin{remark}
%     Deze lemmas zien er willekeurig uit maar zullen later helpen voor een simpeler resultaat. Vanaf nu noemen we
%     $2$de orde differentieoperatoren van volgende vorm $E^2 +c_i E - 1$ simpel.
% \end{remark}

% \begin{lemma}
%     $\sum_{j=0}^{k}a_i^j E^j$ en $E^k + \sum_{j=1}^{k-1}c_i^j E^j + 1$ zijn even moeilijk te factoriseren.
% \end{lemma}

% \begin{proof}
%     Analoog aan lemma ...
% \end{proof}

\subsection{Matrixoperatoren}
Het plan is om lineaire differentievergelijkingen te transformeren door gebruik te maken van
matrixoperatoren-identiteiten naar $1$ste orde lineaire differentievergelijkingen in een grotere groep waar
we een oplossing (\ref{oplo1}) voor hebben. Dit levert een vorm voor de oplossing.

%\cite{elaydi2013introduction}

\begin{theorem}
    $\forall a_i^0,a_i^1,a_i^2 \in (\text{End}(G))_\mathbb{Z}, \exists b_i^0,b_i^1 \in  (\text{End}(G^2))_\mathbb{Z}:$
    $$
        \left[ {
                    \begin{array}{c}
                        a_i^0E^2+a_i^1E+a_i^2 \\
                        0
                    \end{array} } \right]
        =
        \left(
        b_i^0E+b_i^1\right)
        \left[ {
                    \begin{array}{c}
                        E \\
                        1
                    \end{array} } \right]
    $$

\end{theorem}

\begin{proof}
    \begin{align*}
        \left[ {
                    \begin{array}{c}
                        a_i^0E^2+a_i^1E+a_i^2 \\
                        0
                    \end{array} } \right]
         & =
        \left(
        b_i^0E+b_i^1\right)
        \left[ {
        \begin{array}{c}
                        E \\
                        1
                    \end{array} } \right]                              \\
         & =
        b_i^0E \left[ {
                    \begin{array}{c}
                        E \\
                        1
                    \end{array} } \right] +
        b_i^1 \left[ {
        \begin{array}{c}
                        E \\
                        1
                    \end{array} } \right]                              \\
         & =
        b_i^0 \left[ {
                    \begin{array}{c}
                        E^2 \\
                        E
                    \end{array} } \right] +
        b_i^1 \left[ {
        \begin{array}{c}
                        E \\
                        1
                    \end{array} } \right]                              \\
         & =
        \left[ {
                    \begin{array}{cc}
                        b^0_{i11} & b^0_{i12} \\
                        b^0_{i21} & b^0_{i22}
                    \end{array} } \right]
        \left[ {
                    \begin{array}{c}
                        E^2 \\
                        E
                    \end{array} } \right]
        +
        \left[ {
                    \begin{array}{cc}
                        b^1_{i11} & b^1_{i12} \\
                        b^1_{i21} & b^1_{i22}
                    \end{array} } \right]
        \left[ {
        \begin{array}{c}
                        E \\
                        1
                    \end{array} } \right]                              \\
         & =
        \left[ {
        \begin{array}{c}
                        b^0_{i11} E^2 + (b^0_{i12}+b^1_{i11}) E +b^1_{i12} \\
                        b^0_{i21} E^2 + (b^0_{i22}+b^1_{i21}) E +b^1_{i22}
                    \end{array} } \right] \\
    \end{align*}
    Door nu LL en RL te vergelijken of van elkaar af te trekken
    is het eenvoudig te zien dat het volgende een voldoende voorwaarde is:

    $$\left\{
        \begin{aligned}
            a_i^0 & =  b^0_{i11}           \\
            a_i^1 & =  b^0_{i12}+b^1_{i11} \\
            a_i^2 & =  b^1_{i12}
        \end{aligned}
        \right. \text{ en }
        \left\{
        \begin{aligned}
            0 & =  b^0_{i21}           \\
            0 & =  b^0_{i22}+b^1_{i21} \\
            0 & =  b^1_{i22}
        \end{aligned}
        \right. .
    $$
    Wat volgende oplossingen  heeft:
    $$
        b_i^0 = \left[ {
                    \begin{array}{cc}
                        a_i^0 & a_i^1 - b_{i11}^1 \\
                        0     & -b_{i21}^1
                    \end{array} } \right] \text{ en }
        b_i^1 = \left[ {
                    \begin{array}{cc}
                        b_{i11}^1 & a_i^2 \\
                        b_{i21}^1 & 0
                    \end{array} } \right] \text{ met } b_{i11}^1,b_{i21}^1 \in (\text{End}(A))_\mathbb{Z} .
    $$
\end{proof}

% toon ik later denk ik in het algemeen aan dit  is enkel om intuitie op te bouwen voor wat er sebiet komt.
% \begin{remark}
%     Als $a_i^0 , a_i^2 \in (\text{Aut}(A))_\mathbb{Z}$ dan kan ook $b_i^0 ,b_i^1 \in (\text{Aut}(A^2))_\mathbb{Z}$.
% \end{remark}

% \begin{proof}
%     Kies $b_{i11}^1 = 0$  en $b_{i21}^1 = 1$ in vorige formule dan is
%     $\text{det}( b_i^0),\text{det}( b_i^1) \in (\text{Aut}(A))_\mathbb{Z}
%         \Rightarrow b_i^0 ,b_i^1 \in (\text{Aut}(A^2))_\mathbb{Z}$
% \end{proof}

\begin{conjecture}
    $\forall \alpha_i \in (\text{End}(G))_\mathbb{Z}[E]: \text{Deg}(\alpha_i) = k, \exists \beta_i \in (\text{End}(G^2))_\mathbb{Z}[E]:
        \text{Deg}(\beta_i) = k-1:$

    $$
        \left[ {
                    \begin{array}{c}
                        \alpha_i \\
                        0
                    \end{array} } \right]
        =
        \left(
        \beta_i \right)
        \left[ {
                    \begin{array}{c}
                        E \\
                        1
                    \end{array} } \right].
    $$
\end{conjecture}

\begin{proof}
    Analoog aan het vorige bewijs.
\end{proof}

\begin{definition}[companion operator]
    De companion operator van  $\alpha_i \in (\text{End}(G))_\mathbb{Z}[E]$ van graad $k$ en
    coëfficiënten $a^j_i \in(\text{End}(G))_\mathbb{Z} $ is:
    $$
        C(\alpha_i) =
        \left[\begin{array}{ccccc}
                a_i^0  & a_i^1  & a_i^2  & \cdots & a_i^{k-1} \\
                0      & -1     & 0      & \cdots & 0         \\
                0      & 0      & -1     & \cdots & 0         \\
                \vdots & \vdots & \vdots & \ddots & \vdots    \\
                0      & 0      & 0      & 0      & -1
            \end{array}\right]E +
        \left[
            \begin{array}{cc}
                \begin{array}{ccc}
                    0 & 0 & \cdots
                \end{array} & a_{i}^{k}              \\
                I_k                & \begin{array}{c}
                                         \vdots \\
                                         0      \\
                                         0
                                     \end{array}
            \end{array} \right].
    $$

\end{definition}

\begin{remark}
    In de literatuur heeft dit te maken met de companion matrix van een lineaire differentievergelijking. Deze definitie is
    om aan (\ref{relatie companion operator}) te voldoen.
\end{remark}

\begin{example}
    $$
        C(E^2-E-(-1)^i) = \left[
            \begin{array}{cc}
                1 & -1 \\
                0 & -1
            \end{array}
            \right] E +
        \left[
            \begin{array}{cc}
                0 & (-1)^{i+1} \\
                1 & 0
            \end{array}
            \right]
    $$
\end{example}

\begin{theorem}[relatie companion operator]
    \label{relatie companion operator}
    $\forall \alpha_i \in (\text{End}(G))_\mathbb{Z}[E]$ van graad k:
    $$
        \left[\begin{array}{c}
                \alpha_{i} \\
                0          \\
                \vdots     \\
                0
            \end{array}\right]
        =
        \left(
        C(\alpha_i) \right)
        \left[\begin{array}{c}
                E^{k-1} \\
                E^{k-2} \\
                \vdots  \\
                1
            \end{array}\right].
    $$
\end{theorem}
\begin{proof}
    Verificatie.
\end{proof}

\begin{lemma}
    \label{inverteerbare companion}
    Een lineaire operator voldoet aan de uniciteitvoorwaarde als en slechts
    als de companion operator aan de uniciteitvoorwaarde voldoet.
\end{lemma}
\begin{proof}
    Bereken de determinanten van de coëfficiënten.
\end{proof}



\begin{theorem}
    \label{compeq}
    Een differentievergelijking van $k$de orde  van $f_i \in (G)_{\mathbb{Z}}$ kan herschreven
    worden als een $1$ste orde differentievergelijking voor
    $\left[\begin{array}{cccc}
                E^{k-1} &
                E^{k-2} &
                \hdots  &
                1
            \end{array}\right]^T (f_i) = F_i\in (G^k)_{\mathbb{Z}}$.
\end{theorem}

\begin{proof}
    \begin{align*}
        (\alpha_i)(f_i)            & = g_i    \Leftrightarrow    \\
        \left[\begin{array}{cccc}
                      \alpha_{i} &
                      0          &
                      \hdots     &
                      0
                  \end{array}\right]^T (f_i) & =
        \left[\begin{array}{cccc}
                      g_i    &
                      0      &
                      \hdots &
                      0
                  \end{array}\right]^T = G_i      \Leftrightarrow    \\
        \left(
        C(\alpha_i) \right)
        \left[\begin{array}{cccc}
                      E^{k-1} &
                      E^{k-2} &
                      \hdots  &
                      1
                  \end{array}\right]^T(f_i)  & = G_i \Leftrightarrow \\
        \left(
        C(\alpha_i) \right) (F_i)  & = G_i
    \end{align*}
    Van lijn $2$ naar $3$ wordt (\ref{relatie companion operator}) gebruikt.
\end{proof}

\begin{corollary}
    \label{oplok}
    Een vorm voor van de oplossing van een lineaire differentievergelijking die aan de uniciteitvoorwaarde voldoet.
\end{corollary}

\begin{proof}
    Gevolg van (\ref{compeq}),(\ref{inverteerbare companion}) en (\ref{oplo1}).
\end{proof}


% ik wilde eerst f_42 berekenen maar dan werd ik lui
% ik gebruik in dit voorbeeld een specifieke truk voor periodieke lineaire differentie vergelijkingen
% waarvoor ik geen tijd zal hebben :( 
% ik zie dat dit niet perfect is maar ik denk toch beter als geen voorbeeld ...
\begin{example}
    We berekenen hier $f_{2}$ met de oplossingsmethode omschreven in (\ref{oplok}) waar dat:
    $$
        (E^2-E-(-1)^i)(f_i)=1, f_0 = f_1 = 0.
    $$
    Dit zetten we om met (\ref{compeq}) naar:
    $$
        \left(
        \left[
            \begin{array}{cc}
                1 & -1 \\
                0 & -1
            \end{array}
            \right] E +
        \left[
            \begin{array}{cc}
                0 & (-1)^{i+1} \\
                1 & 0
            \end{array}
            \right] \right) (F_i) =
        \left[\begin{array}{c}
                1 \\
                0
            \end{array}\right] = G , F_0 =\left[\begin{array}{c}
                0 \\
                0
            \end{array}\right].
    $$
    De coëfficiënten van deze differentievergelijking zijn inverteerbaar vanwege (\ref{inverteerbare companion})
    daarom kunnen we (\ref{oplo1}) toepassen. Dit geeft het volgende :
    \begin{align*}
        \left[\begin{array}{cc}
                      f_{2} & f_{1}
                  \end{array} \right]^T
        =F_{1} & = H_{1}^{-1}\left( \sum_{j=0}^{0}{\left( S_j^{-1} G \right)}  \right) \\
               & =  H_{1}^{-1} S_0^{-1} G
    \end{align*}

    met
    \begin{align*}
        H_i & = \prod_{j=0}^{i-1}{\left(-\left[
            \begin{array}{cc}
                0 & (-1)^{j+1} \\
                1 & 0
            \end{array}
            \right]^{-1}\left[
            \begin{array}{cc}
                1 & -1 \\
                0 & -1
            \end{array}
        \right] \right)}                        \\
            & =
        \prod_{j=0}^{i-1}{\left(\left[
            \begin{array}{cc}
                0        & -1 \\
                (-1)^{j} & 0
            \end{array}
            \right]\left[
            \begin{array}{cc}
                1 & -1 \\
                0 & -1
            \end{array}
        \right] \right)}                        \\
            & =
        \prod_{j=0}^{i-1}{\left(\left[
            \begin{array}{cc}
                0        & 1          \\
                (-1)^{j} & (-1)^{j+1}
            \end{array}
            \right] \right)}
    \end{align*}
    zodat voor even indices:
    \begin{align*}
        H_{2i} & = \prod_{j=0}^{2i-1}{\left(\left[
            \begin{array}{cc}
                0        & 1          \\
                (-1)^{j} & (-1)^{j+1}
            \end{array}
        \right] \right)}                           \\
               & =
        \prod_{k=0}^{i-1}{\left(\left[
            \begin{array}{cc}
                0  & 1 \\
                -1 & 1
            \end{array}
            \right]\left[
            \begin{array}{cc}
                0 & 1  \\
                1 & -1
            \end{array}
        \right] \right)}                           \\
               & =
        \left[
            \begin{array}{cc}
                1 & -1 \\
                1 & -2
            \end{array} \right]^{i}
    \end{align*}
    en hier uit kunnen de waarden van de oneven indexen gehaald worden
    $$
        H_{2i+1}  =
        \left[
            \begin{array}{cc}
                0 & 1  \\
                1 & -1
            \end{array}
            \right]
        \left[
            \begin{array}{cc}
                1 & -1 \\
                1 & -2
            \end{array} \right]^{i}
    $$
    En $S^{-1}_i$ halen we uit $H_i$:
    \begin{align*}
        S^{-1}_{2i}   & = - \left[
            \begin{array}{cc}
                1 & -1 \\
                1 & -2
            \end{array} \right]^{i}
        \left[
        \begin{array}{cc}
                0  & 1 \\
                -1 & 0
            \end{array} \right]        \\
        S^{-1}_{2i+1} & =
        -\left[
            \begin{array}{cc}
                0 & 1  \\
                1 & -1
            \end{array}
            \right]
        \left[
            \begin{array}{cc}
                1 & -1 \\
                1 & -2
            \end{array} \right]^{i}
        \left[
            \begin{array}{cc}
                0  & 1 \\
                -1 & 0
            \end{array} \right].
    \end{align*}
    Als we hiermee $H_1^{-1}$ en $S_0^{-1}$ uitreken en invullen krijgen we
    $$
        F_1 =
        -\left[ \begin{array}{cc}
                -1 & 1 \\
                -1 & 0
            \end{array}\right]
        \left[ \begin{array}{cc}
                0  & 1 \\
                -1 & 0
            \end{array}\right]
        \left[ \begin{array}{c}
                1 \\
                0
            \end{array}\right]=
        \left[ \begin{array}{c}
                1 \\
                0
            \end{array}\right].
    $$
    Wat wil zeggen dat $f_2 = 1$ wat overeenkomt door gewoon de recursierelatie met $i=0$ te gebruiken.
\end{example}

% \begin{remark}
%     In vorig voorbeeld is het duidelijk eenvoudiger om de recursierelatie zelf te gebruiken.
% \end{remark}

\begin{remark}
    Met (\ref{annihilator}) zouden we nog een andere vorm kunnen maken waar $\Delta^{-1}$ wegvalt door de forceerterm weg te annihileren.
\end{remark}
%  Deze vorm is nuttig voor parallele computaties. maar daarvoor heb ik geen tijd voor :(

\subsection{Operatorfactorisatie}
Het plan is om hogere orde lineaire operatoren te proberen te factoriseren in $1$ste orde lineaire operatoren.
We doen dit eerst voor $2$de orde lineaire operatoren.
We tonen aan dat factoriseren equivalent is met het oplossen van een non-lineaire differentievergelijking.
Als deze geen oplossing heeft, lossen we dit op door de groep groter te maken.


\begin{theorem}[vorm oplossing gefactoriseerde lineaire operator]
    Voor een differentievergelijking die een lineaire operator heeft die aan de uniciteit voorwaarde doet en
    die te schrijven is als een product van $1$ste orde lineaire operatoren kunnen
    we een vorm van de oplossing maken.
\end{theorem}

\begin{proof}
    We gaan dit bewijzen via inductie.\\
    Basis geval: deze stelling voor $1$ste orde differentievergelijkingen, zie (\ref{oplo1}).\\
    Inductie hypothese: deze stelling voor $k-1$de orde differentievergelijkingen.\\
    Inductie stap: deze stelling voor $k$de orde differentievergelijkingen:\\
    $\forall a_i , b_i \in (\text{Aut}(G))_\mathbb{Z},\alpha_i \in (\text{End}(G))_\mathbb{Z}[E]$ van orde $k-1$:
    $$
        (a_iE+b_i)(\alpha_i)(f_i)= g_i.
    $$
    Substitueer $h_i=(\alpha_i)(f_i)$ waarvoor we dankzij de inductie hypothese een vorm van de oplossing kunnen maken. Dit geeft:
    $$
        (a_iE+b_i)(h_i)= g_i.
    $$
    Hiervoor hebben we ook een vorm van de oplossing.
\end{proof}

\begin{theorem}
    \label{gereduceerd factoriseren}
    $\forall a_i^0,a_i^2 \in (\text{Aut}(G))_{\mathbb{Z}},\forall a_i^1 \in (\text{End}(G))_{\mathbb{Z}},
        \exists u_i^j, v_i^j \in (\text{Aut}(G))_{\mathbb{Z}}, \exists c_i \in (\text{End}(G))_{\mathbb{Z}}:$
    $$
        a_{i}^{0} E^{2}+a_{i}^{1} E+a_i^{2} = v_i \left( E^2 + c_i -1 \right) u_i.
    $$
\end{theorem}

\begin{proof}
    \begin{align*}
        a_{i}^{0} E^{2}+a_{i}^{1} E+a_i^{2} & = v_i \left( E^2 + c_i -1 \right) u_i        \\
                                            & = v_iu_{i+1}E^2 + v_ib_i u_{i+1}E - v_i u_i.
    \end{align*}
    Door nu LL en RL te vergelijken of van elkaar af te trekken
    is het eenvoudig te zien dat het volgende een voldoende voorwaarde is:
    $$
        \left\{\begin{aligned}
            a_i^0 & =v_iu_{i+1}      \\
            a_i^1 & =v_i c_i u_{i+1} \\
            a_i^2 & = -v_iu_i
        \end{aligned} \right.
        \Leftarrow
        \left\{\begin{aligned}
            u_i^{-1} & = (-1)^i \prod{(a_i^0)^{-1} a_i^2} \\
            v_i      & = -a_i^2 u_i^{-1}                  \\
            c_i      & =v_i^{-1} a_i^1 u_{i+1}^{-1}
        \end{aligned} \right. .
    $$
\end{proof}

\begin{remark}
    Dit kan ook gedaan worden voor hogere ordes. Dit zorgt ervoor dat we enkel $E^k + \sum_{j=1}^{k-1}c_i^j E^j + 1$ hoeven te factoriseren.
\end{remark}
%orde 2 beginnen voor intuitie
\begin{theorem}[equivalente voorwaarde $2$de orde factorisatie]
    \label{connectie ricatti}
    $\forall c_i \in (\text{End}(G))_{\mathbb{Z}}, \forall H \geq G, \forall n_i \in (\text{Aut}(H))_{\mathbb{Z}}:$
    $$
        E^2 +c_i E -1 = (E - n_i^{-1})(E + n_i) \Leftrightarrow n_{i+1} = c_i + n_i^{-1}.
    $$
\end{theorem}

\begin{proof}
    \begin{align*}
        E^2 +c_i E -1 & = (E - n_i^{-1})(E + n_i)                         \\
                      & = E^2 + E n_i - n_i^{-1} E -1                     \\
                      & = E^2 + (n_{i+1} - n_i^{-1}) E -1 \Leftrightarrow \\
        c_i E         & = (n_{i+1} - n_i^{-1}) E \Leftrightarrow          \\
        c_i           & = n_{i+1} - n_i^{-1} \Leftrightarrow              \\
        n_{i+1}       & = c_i + n_i^{-1}.
    \end{align*}

\end{proof}

\begin{remark}
    Merk op dat we geen uniciteit of existentie van deze factorisatie hebben.
    We hebben enkel een equivalente voorwaarde.
\end{remark}

\begin{definition}[Ricatti differentievergelijking]
    Differentievergelijkingen voor $n_i$ van volgende vorm noemen we van het gereduceerd Ricatti type:
    $$
        n_{i+1} = c_i + n_i^{-1}.
    $$
\end{definition}

\begin{remark}
    Er kan echter ook een connectie in de omgekeerde zin gemaakt worden, namelijk van de Ricatti differentievergelijking naar een 2de orde lineaire operator.
\end{remark}


\begin{theorem}[connectie Ricatti]
    $\forall c_i  \in (\text{End}(G))_{\mathbb{Z}}, \forall x_i \in  (\text{Aut}(G))_{\mathbb{Z}}:$
    $$
        (E^2 + c_i E -1)(x_i) = 0 \Leftrightarrow n_i := -x_{i+1}x_i^{-1} \text{ en } n_{i+1} = c_i + n_i^{-1}.
    $$
\end{theorem}

\begin{proof}
    Dit was te vermoeden want
    $(\Leftarrow):$
    \begin{align*}
        (E^2 + c_i E -1)(x_i)        & = 0                 \\
        (E - n_i^{-1})(E + n_i)(x_i) & = 0 \Leftarrow      \\
        (E + n_i)(x_i)               & = 0 \Leftrightarrow \\
        n_i                          & = -x_{i+1}x_i^{-1}.
    \end{align*}
    $(\Leftrightarrow)$ is verificatie:
    \begin{align*}
        0                     & = (E^2 + c_i E -1)(x_i)      \Leftrightarrow \\
        -x_{i+2}              & =  c_i x_{i+1} - x_i  \Leftrightarrow        \\
        -x_{i+2} x_{i+1}^{-1} & = c_i - x_i x_{i+1}^{-1}
    \end{align*}
    vervang $n_i = - x_i x_{i+1}^{-1}$ en dat toont het gevraagde.
\end{proof}

\begin{remark}
    De Ricatti differentievergelijking heeft een connectie met continuerende breuken.
    In \cite{https://doi.org/10.48550/arxiv.2103.03554} worden continuerende breuken-identiteiten bewezen via het factoriseren van
    $2$de orde lineaire operator.
\end{remark}


\begin{theorem}
    In een lichaam kunnen we voor alle $2$de orde operatoren een factorisatie vinden in een groter lichaam.
    $\forall c_i \in (F)_{\mathbb{Z}}, \exists n_i \in (F(X)\setminus F)_{\mathbb{Z}}:$
    $$
        E^2 +c_i E -1 = (E - n_i^{-1})(E + n_i) \Leftrightarrow n_{i+1} = c_i + n_i^{-1}.
    $$
\end{theorem}

\begin{proof}
    Construeer $n_i$ door de voorwaarden $n_{i+1} = c_i + n_i^{-1}$ en
    $n_0\in (F(X)\setminus F)_{\mathbb{Z}}$.
    $n_i$ is goed gedefinieerd ($n_i^{-1}$ en $(n_{i} - c_{i-1})^{-1}$ bestaat) want $n_i \in (F(X)\setminus F)_{\mathbb{Z}}$ en
    dat tonen we aan met inductie.\\
    Basis geval:
    $$
        n_0  \in F(X)\setminus F.
    $$
    Inductie hypothese:
    $$
        \forall |j|<k: n_j \in F(X)\setminus F.
    $$
    Inductie stap:\\
    Stel dat $n_{k} \in F$ dan:
    \begin{align*}
        F \ni n_{k} -c_{k-1} & = n_{k-1}^{-1} \in F(X)\setminus F
    \end{align*}
    Wat een contradictie is.\\
    $$n_{-k} = (n_{-k+1} - c_{k})^{-1} \in F(X)\setminus F \Leftrightarrow n_{-k+1} - c_{k} \in F(X)\setminus F.$$
    Stel dat $ n_{-k+1} - c_{k} \in F \Leftrightarrow n_{-k+1} \in F$ wat een contradictie is.






\end{proof}

\begin{example}
    Neem bijvoorbeeld $E^2+E+1$ welke niet splitst in $\mathbb{F}_2$, maar wel bijvoorbeeld in $\mathbb{F}_2(X)$ en
    $\frac{\mathbb{F}_2[X]}{(X^2+X+1)}$.In volgende tafel staan voorbeelden van oplossingen van de Ricatti vergelijking.
    \begin{table}[ht]
        \centering
        \begin{tabular}{|c||c|c|c|c|c|c|c|c|c|c|} \hline
                                                & ... & $n_{-2}$        & $n_{-1}$        & $n_{0}$ & $n_{1}$         & $n_{2}$         & $n_{3}$ & $n_{4}$         & $n_{5}$         & ... \\ \hline \hline
            $\frac{\mathbb{F}_2[X]}{(X^2+X+1)}$ & ... & $X$             & $X$             & $X$     & $X$             & $X$             & $X$     & $X$             & $X$             & ... \\ \hline
            $\mathbb{F}_2(X)$                   & ... & $\frac{1+X}{X}$ & $\frac{1}{1+X}$ & $X$     & $\frac{1+X}{X}$ & $\frac{1}{1+X}$ & $X$     & $\frac{1+X}{X}$ & $\frac{1}{1+X}$ & ... \\ \hline
        \end{tabular}
    \end{table}
    De corresponderende homogene oplossingen  in deze lichaam extensies zijn $\prod{n_i}$.
    In principe hadden we ook eerst inverteerbare homogene oplossingen kunnen zoeken in deze lichaam extensies.
\end{example}


\begin{conjecture}[equivalente factorisatie voorwaarden]
    $\forall c_i^j ,d_i^j ,n_i \in (\text{End}(G))_{\mathbb{Z}}:$
    \begin{align*}
         & E^k + \sum_{j=1}^{k-1}c_i^j E^j + 1  = (\sum_{j=0}^{k-1}d_i^j E^j)(E+n_i) \Leftrightarrow                      \\
         & \left\{\begin{aligned}
                      c^j_i     & = d^j_i n_{i+j}+d^{j-1}_i \text{ met } d^{k-1}_i=1, d^0_i = n^{-1}_i                        \\
                      n_{i+k-1} & = c^{k-1}_i-n^{-1}_{i+k-2}(c^{k-2}_i-n^{-1}_{i+k-3}(... -n^{-1}_{i+1}(c^{1}_i-n^{-1}_{i})))
                  \end{aligned} \right.
    \end{align*}

\end{conjecture}

\begin{proof}
    Geen bewijs.
\end{proof}

\begin{remark}
    Merk op als we $k=2$ kiezen in vorige vergelijkingen dat we dan terug de Ricatti differentievergelijkingen krijgen.
    $$
        k =2  : c^1_i - n_{i}^{-1}= n_{i+1}, k=3 : c^2_i - n_{i+1}^{-1} (c^1_i - n_{i}^{-1}) = n_{i+2}
    $$
\end{remark}

\begin{conjecture}
    Er bestaat zoals in het geval $k=2$ ook een connectie terug naar de lineaire operator.
\end{conjecture}

\begin{proof}
    Geen bewijs. Via dezelfde substitutie om dezelfde reden.
\end{proof}

\begin{conjecture}
    Zoals in het geval $k=2$ kunnen we altijd een oplossing $n_i \in F(X_0,X_1,...,X_{k-1}) \setminus F$ vinden.
\end{conjecture}
\begin{proof}
    Geen bewijs.
\end{proof}

\begin{example}
    Neem bijvoorbeeld $E^3+E^2+E+1$ welke niet splitst in $\mathbb{Q}$, maar wel bijvoorbeeld in $\mathbb{Q}(X,Y)$ en
    $\frac{\mathbb{Q}[X]}{(X^3+X^2+X+1)}$.
    \begin{table}[ht]
        \centering
        \begin{tabular}{|c||c|c|c|c|c|c|c|c|} \hline
                                                  & ... & $n_{0}$ & $n_{1}$ & $n_{2}$                   & $n_{3}$                 & $n_{4}$ & $n_{5}$ & ... \\ \hline \hline
            $\frac{\mathbb{Q}[X]}{(X^3+X^2+X+1)}$ & ... & $X$     & $X$     & $X$                       & $X$                     & $X$     & $X$     & ... \\ \hline
            $\mathbb{Q}(X,Y)$                     & ... & $X$     & $Y$     & $\frac{X Y - X + 1}{X Y}$ & $\frac{1}{X Y - X + 1}$ & $X$     & $Y$     & ... \\ \hline
        \end{tabular}
    \end{table}
\end{example}
% ik wilde graag nog een voorbeeld laten zien waar ik ring extensies gebruik om bepaalde operatoren te laten factoriseren
% maar daar heb ik geen tijd voor.

\begin{remark}
    In het algemeen zijn oplossingen van de equivalente factorisatie voorwaarden niet zo ''goed gedragend''.
\end{remark}

\begin{example}
    Neem nu $E^2 + 42 E - 1$. Daar is een oplossing voor de Ricatti vergelijking in $\mathbb{Q}(X):$
    \begin{table}[ht]
        \centering
        \begin{tabular}{|c||c|c|c|c|c|c|c|} \hline
                            & ... & $n_{0}$ & $n_{1}$            & $n_{2}$                        & $n_{3}$                              & $n_{4}$                                   & ... \\ \hline \hline
            $\mathbb{Q}(X)$ & ... & $X$     & $42 + \frac{1}{X}$ & $\frac{1765 X + 42}{42 X + 1}$ & $\frac{74172 X + 1765}{1765 X + 42}$ & $\frac{3116989 X + 74172}{74172 X + 1765}
            $               & ...                                                                                                                                                          \\ \hline
        \end{tabular}
    \end{table}
\end{example}





% Dingens uit vorige versies
% %wat ik ook kan doen is factoriseren in commuterende termen

% \begin{theorem}
%     Lineaire differentieoperatoren factoriseren is even moeilijk als een oplossing
%     vinden van een aantal lineaire homogene differentievergelijkingen.
% \end{theorem}

% \begin{proof}
%     Het is voldoende om $E^k + \sum_{j=1}^{k-1}c_i^j E^j + 1$ te factoriseren voor $k =1$ is het resultaat triviaal.
%     Voor $k \geq 2$ gok volgende vorm
%     \begin{align*}
%         \sum_{j=0}^{k}c_i^j E^j & = (\sum_{j=0}^{k-1}d_i^j E^j)(E+n_i)              \\
%                                 & = \sum_{j=1}^{k}(d^j_i n_{i+j}+d^{j-1}_i)  E^j +1
%     \end{align*}
%     Beide kanten vergelijken geeft:
%     $$
%         c^j_i = d^j_i n_{i+j}+d^{j-1}_i \text{ met } d^{k-1}_i=1, d^0_i = n^{-1}_i
%     $$
%     Wat een 1ste orde lineaire differentievergelijking met beginvoorwaardes is voor rijen met als onbekende $d^j_i$
%     waarvoor explciet een unieke oplossing voor te vinden is voor elke $n_i$. Dus het is voldoende om $n_i$ te vinden.
%     Om $n_i$ te vinden gaan we recursief beginnende van $j=0$, $d^j_i$ substitueren tot en met $j=k-1$. Dit geeft de volgende
%     $(k-1)$orde differentievergelijking voor $n_i$.
%     $$
%         c^{k-1}_i-n^{-1}_{i+k-2}(c^{k-2}_i-n^{-1}_{i+k-3}(... -n^{-1}_{i+1}(c^{1}_i-n^{-1}_{i}))) = n_{i+k-1}
%     $$

%     $$
%         c^{k-1}_i+ \sum_{j = 2}^{k} \frac{c_i^{k-j}}{\prod_{l=2}^j -n_{i+k-l}} = n_{i+k-1}
%     $$


%     Hier zijn een aantal voorbeelden:
%     $$
%         k =2 \text{ (Riccati)} : c^1_i - \frac{1}{n_{i}}= n_{i+1}, k=3:c^{2}_i - \frac{c^{1}_i - \frac{1}{n_{i}}}{n_{i+1}}= n_{i+2}
%     $$

%     Hier is wat motivatie voor volgende substitutie. Stel dat we een homogene oplossing $f^H_i$ hebben dan is het niet
%     moeilijk om te vermoeden dat deze ook een homogene oplossing is van de laatste factor van één van de mogelijke factorisaties:
%     \begin{align*}
%         (E + n_i)(f^H_i)    & = 0 \Leftrightarrow        \\
%         f^H_{i+1}+ n_if^H_i & = 0 \Leftrightarrow        \\
%         n_i                 & = -\frac{f^H_{i+1}}{f^H_i}
%     \end{align*}
%     Dit motiveert $n_i = -\frac{f_{i+1}}{f_i}$ en dit geeft %WP: motiveerT (gefixt)

%     \begin{align*}
%         c^{k-1}_i+ \sum_{j = 2}^{k} \frac{c_i^{k-j}}{\prod_{l=2}^j \frac{f_{i+k-l+1}}{f_{i+k-l}} } & = -\frac{f_{i+k}}{f_{i+k-1}} \Leftrightarrow  \\
%         c^{k-1}_i+ \sum_{j = 2}^{k} \frac{f_{i+k-j}c_i^{k-j}}{f_{i+k-1}}                           & = -\frac{f_{i+k}}{f_{i+k-1}}  \Leftrightarrow \\
%         f_{i+k} + f_{i+k-1} c^{k-1}_i+ \sum_{j = 2}^{k} f_{i+k-j}c_i^{k-j}                         & = 0 \Leftrightarrow                           \\
%         (\sum_{j=0}^{k}c_i^j E^j)(f_i)                                                             & = 0
%     \end{align*}

%     Dit toont dat een factor uit $\sum_{j=0}^{k}c_i^j E^j$ factoriseren even moeilijk is als een oplossing vinden van het corrensponderende homogene
%     differentievergelijking. Dus $\sum_{j=0}^{k}c_i^j E^j$ factoriseren reduceert zich dan naar het reduceren van een lineaire differentieoperator
%     van een orde lager waar dan inductief hetzelfde argument kan op gebruikt worden.

% \end{proof}

% % ok dit kan ik waarschijnlijk ergens anders zetten maar dat zie ik daarna wel
% \begin{definition}[lineair onafhankelijk]
%     $\{a_i^j \in (A_0)_{\mathbb{Z}}\}_{j \in J}$ is $B$-lineair onafhankelijk $\Leftrightarrow$
%     $$
%         \forall b_j \in B \leq \text{End}(A): \sum_{j}b_j a_i^j = 0 \Rightarrow   b_j = 0 .
%     $$
% \end{definition}

% in vorige versie was volgend lemma in 2 stukken
% \begin{lemma}
%     Lineair onafhankelijkheid wordt behouden onder $z_i \in (Z(\text{Aut}(A)))_\mathbb{Z}$.
% \end{lemma}

% \begin{proof}
%     Stel  $\{a_i^j \in (A_0)_{\mathbb{Z}}\}_{j \in J}$ is lineair onafhankelijk dan is
%     $$
%         \forall b_j \in B: \sum_{j}b_j z_i a_i^j = 0 \Rightarrow z_i \sum_{j}b_j a_i^j = 0
%         \Rightarrow \sum_{j}b_j a_i^j = 0  \Rightarrow  b_j = 0.
%     $$
% \end{proof}

% \begin{lemma}
%     Lineair onafhankelijkheid wordt behouden onder de $\sum$ operator.
% \end{lemma}

% \begin{proof}
%     Stel  $\{a_i^j \in (A_0)_{\mathbb{Z}}\}_{j \in J}$ is lineair onafhankelijk dan is

%     \begin{align*}
%         \forall b_j,b \in \text{End}(A):0 & = \sum_{j}b_j \sum(a_i^j)             \\
%                                           & = \sum(\sum_{j}b_j a_i^j) \Rightarrow \\
%         0                                 & = \Delta \sum(\sum_{j}b_j a_i^j)      \\
%                                           & = \sum_{j}b_j a_i^j \Rightarrow       \\
%         b_j                               & = 0
%     \end{align*}
% \end{proof}


% extra dingens over lineair onafhankelijkheid
% \begin{lemma}
%     Als $\{a_i^j \in (A_0)_{\mathbb{Z}}\}_{j \in J}$ $B$-lineair onafhankelijk is dan
%     is $\{ \Delta^{-1}(z_i a_i^j)\}_{j \in J} \cup \{k\} $
%     met $k \in (A_0)_{\mathbb{Z}}$ en $z_i \in (Z(B)^*)_\mathbb{Z}$  ook $B$-lineair onafhankelijk.
% \end{lemma}

% \begin{proof}
%     \begin{align*}
%         \forall b_j,b \in B: 0 & = \sum_{j }b_j \Delta^{-1}(z_i a_i^j) + bk \Leftrightarrow     \\
%         -bk                    & = \Delta^{-1}(\sum_{j }b_j z_i a_i^j)  \Rightarrow             \\
%         \Delta(-bk)            & = \Delta \Delta^{-1}(\sum_{j }b_j z_i a_i^j)   \Leftrightarrow \\
%         0                      & = \sum_{j }b_j z_i a_i^j \Leftrightarrow                       \\
%         0                      & = z_i \sum_{j }b_j  a_i^j\Leftrightarrow                       \\
%         0                      & = \sum_{j }b_j  a_i^j \Rightarrow                              \\
%         b_j                    & = 0 \Leftrightarrow                                            \\
%         0                      & = bk \Leftrightarrow                                           \\
%         b                      & = 0
%     \end{align*}
% \end{proof}

% \begin{theorem}
%     De oplossingen voor een lineaire homogene $k$ orde differentievergelijking vormen een $k$ dimensionale vectorruimte.
% \end{theorem}

% \begin{proof}
%     Standaard inductie bewijs op de orde $k$. Basisgeval $k=0$ is triviaal. Stel dat het gevraagde geldt voor $k <m$. De algemene
%     vorm voor een lineaire homogene $m$ orde differentievergelijking is:
%     $$
%         (\sum_{j=0}^{m}c_i^j E^j)(f_i) = 0
%     $$
%     Dankzij stelling ... equivalent met
%     $$
%         (\sum_{j=0}^{m-1}d_i^j E^j)(E+n_i)u^{-1}_i(f_i) = 0
%     $$
%     Gebruik volgende substitutie:
%     \begin{align*}
%         g_i & = (E+n_i)u^{-1}_i(f_i) \Leftrightarrow                                                 \\
%         f_i & =  \frac{u_i}{h_i} \left( \sum {\left( \frac{g_i}{s_i} \right)} + u^{-1}_0 f_0 \right)
%     \end{align*}
%     Dit geeft
%     $$
%         (\sum_{j=0}^{m-1}d_i^j E^j)(g_i)) = 0
%     $$
%     Dankzij de inductiehypothese is de oplossingruimte van $g_i$ een $(m-1)$ dimensionale vectorruimte. Dankzij vorige lemmas en
%     terug substitueren volgt het gevraagde.
% \end{proof}


%WP: ik denk toch dat je beter de aanpak van het andere boek (Elaybi) zou volgen, 
%want deze tekst is bij momenten erg verwarrend. Het is me onder meer ook niet duidelijk 
%waarom je je indexverzameling per se in \mathbb{Z} wil nemen. Differentievergelijkingen hebben 
%altijd een aspect dat dit resultaten van metingen zijn in functie van de tijd, en er is niet zoiets als negatieve tijd.
% Ik denk ook dat de tekst gebaat zou zijn met meer NUMERIEKE voorbeelden, dus concreet differentievergelijkingen met getalletjes in, 
%waarvan je dan de oplossing berekent. Uiteraard mag je daar voorbeelden nemen waarbij de getalletjes mooi uitkomen.
% Bijvoorbeeld een differentievergelijking van de gedaante y_{n+2}-4y_{n+1}-5y_n = 5^n of zoiets, 
%wat een mooi voorbeeld is van een vergelijking waarbij je geen annihilator maar een hyperannihilator moet gebruiken.


\newpage
\printbibliography

\end{document}